\documentclass[pdf, 10pt, unicode]{beamer}

\usepackage[T2A]{fontenc}
%\usepackage[cp1251]{inputenc}
\usepackage[utf8]{inputenc}
\usepackage[russian,english]{babel}
\usepackage{minted}
\usemintedstyle{perldoc}
\usepackage{graphicx}
\usepackage{amssymb}
\usepackage{amsthm}
\usepackage{amsmath}
\usepackage{tabularx}
\usepackage{multicol}
\usepackage{multirow}
\usepackage{calrsfs}
\usepackage{listings}
\usepackage{color}

\usepackage{array}
\newcolumntype{L}[1]{>{\raggedright\let\newline\\\arraybackslash\hspace{0pt}}m{#1}}
\newcolumntype{C}[1]{>{\centering\let\newline\\\arraybackslash\hspace{0pt}}m{#1}}
\newcolumntype{R}[1]{>{\raggedleft\let\newline\\\arraybackslash\hspace{0pt}}m{#1}}

% \lstset{basicstyle=\fontsize{7pt}{1em}\ttfamily,
%         keywordstyle=\fontsize{7pt}{1em}\color{MOOCBlue}\bfseries,
%         commentstyle=\fontsize{7pt}{1em}\ttfamily\color{MOOCGreen}}


\usepackage{caption}
\usepackage{subcaption}

% \lstset{language=C++,
%   basicstyle=\footnotesize\itshape,
%   keywordstyle=\color{black}\bfseries\underbar,
%   % underlined bold black keywords
%   identifierstyle={\color{black}},%
%   commentstyle=\color{brown}, % brown comments
%   stringstyle=\color{blue}\ttfamily, % typewriter type for strings, blue
%   showstringspaces=false
% }

\definecolor{MOOCGrey}{RGB}{230,230,230}
\definecolor{MOOCBlue}{RGB}{38,57,200}
\definecolor{MOOCGreen}{RGB}{38,130,63}

\lstset{extendedchars=true}
\lstset{language=C++, basicstyle=\small\ttfamily,  keywordstyle=\small\color{MOOCBlue}\bfseries,
        commentstyle=\small\ttfamily\color{MOOCGreen}, showstringspaces=false,
        extendedchars=true, texcl=true, frame=single,
        morekeywords={size_t}}



\newenvironment{changemargin}[2]{%
\begin{list}{}{%
\setlength{\topsep}{0pt}%
\setlength{\leftmargin}{#1}%
\setlength{\rightmargin}{#2}%
\setlength{\listparindent}{\parindent}%
\setlength{\itemindent}{\parindent}%
\setlength{\parsep}{\parskip}%
}%
\item[]}{\end{list}}


\usetheme{Copenhagen}
\usepackage{color}

\usecolortheme[RGB={50,100,200}]{structure}

\setbeamertemplate{navigation symbols}{}
\setbeamertemplate{itemize
item}{\scriptsize\raise1.25pt\hbox{\donotcoloroutermaths$\blacktriangleright$}}

\setbeamertemplate{frametitle continuation}[from second]

\title{Воспоминания об указателях.}
\author{Евгений Линский}
\date{}
\usefonttheme[onlymath]{serif}

\setbeamerfont{institute}{size=\normalsize}

\setbeamertemplate{footline} {
  \hbox{%
  \begin{beamercolorbox}[wd=0.08\paperwidth,ht=2.5ex,dp=0ex,left]{title in head/foot}%
  \end{beamercolorbox}%
  \begin{beamercolorbox}[wd=0.46\paperwidth,ht=2.5ex,dp=1ex,center]{title in head/foot}%
    \usebeamerfont{title in head/foot}{C++}%\insertshortauthor %\insertshorttitle
  \end{beamercolorbox}%
  \begin{beamercolorbox}[wd=0.46\paperwidth,ht=2.5ex,dp=1ex,right]{date in head/foot}%
    \usebeamerfont{date in head/foot}
    \insertframenumber{} / \inserttotalframenumber{}\hspace*{2ex}
  \end{beamercolorbox}}%
  \vskip0pt%
}

\sloppy

\begin{document}
\begin{frame}
  \vspace{1cm}
  \large
  \maketitle
  \thispagestyle{empty}
  \vspace{1cm}
  \date{}
\end{frame}

\begin{frame}[fragile]{{\tt Типы}}

\begin{tabular}{lll}
Название & Кол-во байт & Диапазон \\
типа     & для хранения & \\
\hline
char  & 1 & $-2^{7}..2^{7}-1$ \\
short & 2 & $-2^{15}..2^{15}-1$ \\
int   & 4 & $-2^{31}..2^{31}-1$ \\
long  & 4 & $-2^{31}..2^{31}-1$ \\
long long & 8 & $-2^{63}..2^{63}-1$ \\
unsigned char  & 1 & $0..2^{8}-1$ \\
unsigned short & 2 & $0..2^{16}-1$ \\
unsigned int   & 4 & $0..2^{32}-1$ \\
unsigned long  & 4 & $0..2^{32}-1$ \\
unsigned long long & 8 & $0..2^{64}-1$ \\
\hline
float  & 4 & $1,4 \cdot 10^{-45}.. 3.4 \cdot 10^{38}$ \\
double & 8 & $4,94 \cdot 10^{-324}.. 1.79 \cdot 10^{308}$ \\
\end{tabular}

В чем подвох?
\end{frame}

\begin{frame}[fragile]{{\tt Зависимость от платформы}}
\begin{enumerate}
  \item На самом деле размеры типов зависят от платформы (процессор, ОС, компилятор)
  \item int --- ``естественный'' тип (компьютеру проще работать: ширина регистров, особенности набора инструкций)
  \item На самом деле, например: $ sizeof(short) \leq sizeof(int) \leq sizeof(long) $
  \item \emph{sizeof} --- оператор языка (не функция), во время компиляции заменяется на размер типа
\end{enumerate}
\end{frame}

\begin{frame}[fragile]{{\tt Массивы}}
Одномерные:
\begin{minted}[linenos=true]{cpp}
  int array[10]; // размер  10*sizeof(int)
  //Инициализация:
  int array[10] = {0, 1, 2, 3, 4, 5, 6, 7, 8, 9};
  int array[10] = {0}; // обнулить
  int array[] = {0, 1, 2, 3, 4, 5, 6, 7, 8, 9};
  //для типа char:
  char array[] = {0, 1, 2, 3, 4, 5, 6, 7, 8, 9};
  char array[] = {'H', 'e', 'l', 'l', 'o'};
  char array[] = "Hello"; // размер?
\end{minted}

\begin{enumerate}
\uncover<2->{\item Рзамер 6 (завершающий 0)!}
\end{enumerate}

\end{frame}

\begin{frame}[fragile]{{\tt Указатели}}
\begin{enumerate}
  \item Указатель (pointer) --- число, адрес (т.е. смещение от начала) соответствующего элемента в памяти
  \item int* p; --- указатель на ячейку, в которой хранится int (в `p' будет хранится адрес)
  \item Количество байт для хранения указателя зависит от архитектуры компьютера (на x86 сейчас --- 64 бита)
    \item sizeof(int*) == sizeof(char*) == sizeof(double*) etc
\end{enumerate}
\end{frame}

\begin{frame}[fragile]{{\tt Указатели}}
\begin{minted}[linenos=true]{cpp}
  int a = 42;
  int *p = &a; // & -- взять адрес a
  int b = *p; // взять значение по адресу p (разыменовать)
  printf("%p", p); // вывести адрес
\end{minted}
\end{frame}

\begin{frame}[fragile]{{\tt Указатели}}
Сдвиг зависит от типа объекта, на который указывает указатель.
\begin{minted}[linenos=true]{cpp}
  int array[5] = {1, 2, 3, 4, 5};
  char str[] = "hello";
  int *pi = array; // pi = &array[0]
  char *pc = str; // pc = &str[0]
  pi += 1; // сдвиг адреса на sizeof(int)
  pc += 1; // сдвиг адреса на sizeof(char)
\end{minted}

array[i] == i[array]:
\begin{minted}[linenos=true]{cpp}
array[i] --> *(array + i)
i[array] --> *(i + array)
\end{minted}

\end{frame}

\begin{frame}[fragile]{{\tt Различие между разными видами указателей}}
\begin{minted}[linenos=true]{cpp}
char str[4];
char *pc = &c[0]; // &c[0] – адрес нулевого элемента массива
                  // или адрес массива
int *pi = pc; // C -- ok, C++ -- error (разные типы)
int *pi = (int*)pc; // C -- ok, C++ -- ok
\end{minted}

\begin{minted}[linenos=true]{cpp}
char array[10];
char *p;
p = array; // в p передается адрес массива
           // (адрес нулевого элемента)
array = p; // это ошибка
\end{minted}
\end{frame}

\begin{frame}[fragile]{{\tt Применение - 1}}
\begin{minted}[linenos=true]{cpp}
void swap(double a, double b){
  double tmp = a;
  a = b;
  b = tmp;
}
int main() {
  double c = 3; double d = 4;
  swap(c, d);
  return 0;
}
\end{minted}
\begin{enumerate}
\uncover<2->{\item Ничего не получится.}
\uncover<3->{\item Функция работает с копиями параметров (a и b поменяются, c и d нет).}
\end{enumerate}

\end{frame}

\begin{frame}[fragile]{{\tt Применение - 1}}
\begin{minted}[linenos=true]{cpp}
void swap(double *pa, double *pb){
  double tmp = *pa;
  *pa = *pb;
  *pb = tmp;
}
int main() {
  double c = 3; double d = 4;
  swap(&c, &d);
  return 0;
}
\end{minted}

\end{frame}

\begin{frame}[fragile]{{\tt Применение - 2}}
Передать в функцию большой объект и не копировать его!
\begin{minted}[linenos=true]{cpp}
char str[] = "Hello";
int l = strlen(str);
\end{minted}

\end{frame}

\begin{frame}[fragile]{{\tt Применение - 3}}
\begin{tabular}{ll}
\begin{lstlisting}
int strlen(char* ptr){
  int len = 0;
  while (ptr[len] != '\0'){
    ++len;
  }
  return len;
}
\end{lstlisting}
&
\begin{lstlisting}
int strlen(char* ptr){
  char* p = ptr;
  while (*p != '\0'){
    ++p;
  }
  return p - ptr;
}
\end{lstlisting}
\end{tabular}

\begin{enumerate}
\uncover<2->{\item ptr[len] --> *(ptr+len) одно сложение!}
\uncover<3->{\item '\textbackslash0' --- символ с кодом 0.}
\uncover<4->{\item (p - ptr) --- длина строки (складывать указатели нельзя) .}
\end{enumerate}

\end{frame}

\begin{frame}[fragile]{{\tt const у указателя}}
const защищает то, что \emph{перед} ним.
\begin{minted}[linenos=true]{cpp}
char s1[] = "hello";
char s2[] = "bye";
char const * p1 = s1;
p1[0] = 'a'; // compilation error
p1 = s2; // ok
char * const p2 = s1;
p2[0] = 'a'; // ok
p2 = s2; // compilation error
char const * const p3 = s1;
\end{minted}
Но можно и так:
\begin{minted}[linenos=true]{cpp}
const char * p1; // equal to char const * p1;
\end{minted}
\end{frame}


\begin{frame}[fragile]{{\tt const у указателя}}
\begin{minted}[linenos=true]{cpp}
size_t strlen(const char * s);
int main() {
  char str[] = "Hello";
  site_t s = strlen(str);
}
\end{minted}
\begin{enumerate}
  \uncover<1->{\item Что хотел сказать программист?}
  \uncover<2->{\item Функция \emph{strlen} не изменяет свой аргумент. Например, программист в main может не делать копию \emph{str} перед вызовом \emph{strlen}.}
\end{enumerate}
\end{frame}

\begin{frame}[fragile]{{\tt const у указателя}}
\begin{minted}[linenos=true]{cpp}
void strange(const char * s) {
  char* str = (char*)s;
  str[0] = 'A'
}
\end{minted}

\begin{minted}[linenos=true]{cpp}
char s[] = "Hello";
strange(s);
\end{minted}
Все ли хорошо?

\begin{minted}[linenos=true]{cpp}
const char* const s = "Hello";
strange(s);
\end{minted}
Все ли хорошо?

\end{frame}

\begin{frame}[fragile]{{\tt strchr}}

Задача: написать функцию, которая возвращает адрес первого вхождения символа в строку.

\begin{lstlisting}
char* strchr(const char* s, char ch) {
  while (*s != 0) {
    if (*s == ch)
      return s;
    s++;
  }
  return NULL;
}
\end{lstlisting}

\begin{itemize}
  \item Если функция ничего не нашла, то она возвращает нулевой адрес.
  \item В стандартной библиотеке языка C есть макрос \# define NULL 0 для более явного обозначения нулевого адреса.
\end{itemize}

\end{frame}

\begin{frame}[fragile]{{\tt strchr}}

Задача: найти первое и второе вхождение

\begin{lstlisting}
char s[] = "Hello, world!";
char* p1;
char* p2;

p1 = strchr(s, 'o');
p2 = strchr(p1, 'o');

\end{lstlisting}

Что не так?
\begin{enumerate}
  \uncover<2->{\item Есть ли нет первого вхождения, то будет ошибка (p1 --- NULL)}
  \uncover<3->{\item Если есть первое вхождение, найдет его повторно}
\end{enumerate}

\end{frame}

\begin{frame}[fragile]{{\tt strchr}}

Задача: найти первое и второе вхождение

\begin{lstlisting}
char s[] = "Hello, world!";
char* p1;
char* p2;

p1 = strchr(s, 'o');
if (p1 != NULL)
  p2 = strchr(p1 + 1, 'o');

\end{lstlisting}

\end{frame}


\begin{frame}[fragile]{{\tt strcpy}}

\begin{lstlisting}
char* strcpy(char* to, const char* from) {
  char *save = to;
  for (; (*to = *from) != '\0'; ++from, ++to);
  return(save);
}
\end{lstlisting}
Скопируем ли мы последний символ 0?

\end{frame}

\begin{frame}[fragile]{{\tt strcat}}

\begin{lstlisting}
char* strcat(char *dest, const char *src) {
  strcpy (dest + strlen(dest), src);
  return dest;
}
\end{lstlisting}

\end{frame}


\begin{frame}[fragile]{{\tt inplace reverse}}

\begin{lstlisting}
void strrev(char* head) {
  char *tail = head;
  while(*tail) ++tail;
  --tail;
  for( ; head < tail; ++head, --tail) {
      char h = *head, t = *tail;
      *head = t;
      *tail = h;
  }
}
\end{lstlisting}

\end{frame}


\begin{frame}[fragile]{{\tt strtok}}

\begin{lstlisting}
char* strtok( char* str, const char* delim );
\end{lstlisting}

\begin{lstlisting}
    char input[] = "one + two * (three - four)!";
    const char* delimiters = "! +- (*)";
    char* token = strtok(input, delimiters);
    while (token) {
        printf("%s\n", token);
        token = strtok(NULL, delimiters);
    }
\end{lstlisting}
Как это работает? Есть ли подсказка в объявлении функции?

\end{frame}

\end{document}




