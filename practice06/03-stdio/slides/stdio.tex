\documentclass[pdf, 10pt, unicode]{beamer}

\usepackage[T2A]{fontenc}
%\usepackage[cp1251]{inputenc}
\usepackage[utf8]{inputenc}
\usepackage[russian,english]{babel}
\usepackage{minted}
\usemintedstyle{perldoc}
\usepackage{graphicx}
\usepackage{amssymb}
\usepackage{amsthm}
\usepackage{amsmath}
\usepackage{tabularx}
\usepackage{multicol}
\usepackage{multirow}
\usepackage{calrsfs}
\usepackage{listings}
\usepackage{color}

\usepackage{array}
\newcolumntype{L}[1]{>{\raggedright\let\newline\\\arraybackslash\hspace{0pt}}m{#1}}
\newcolumntype{C}[1]{>{\centering\let\newline\\\arraybackslash\hspace{0pt}}m{#1}}
\newcolumntype{R}[1]{>{\raggedleft\let\newline\\\arraybackslash\hspace{0pt}}m{#1}}

% \lstset{basicstyle=\fontsize{7pt}{1em}\ttfamily,
%         keywordstyle=\fontsize{7pt}{1em}\color{MOOCBlue}\bfseries,
%         commentstyle=\fontsize{7pt}{1em}\ttfamily\color{MOOCGreen}}


\usepackage{caption}
\usepackage{subcaption}

% \lstset{language=C++,
%   basicstyle=\footnotesize\itshape,
%   keywordstyle=\color{black}\bfseries\underbar,
%   % underlined bold black keywords
%   identifierstyle={\color{black}},%
%   commentstyle=\color{brown}, % brown comments
%   stringstyle=\color{blue}\ttfamily, % typewriter type for strings, blue
%   showstringspaces=false
% }

\definecolor{MOOCGrey}{RGB}{230,230,230}
\definecolor{MOOCBlue}{RGB}{38,57,200}
\definecolor{MOOCGreen}{RGB}{38,130,63}

\lstset{extendedchars=true}
\lstset{language=C++, basicstyle=\small\ttfamily,  keywordstyle=\small\color{MOOCBlue}\bfseries,
        commentstyle=\small\ttfamily\color{MOOCGreen}, showstringspaces=false,
        extendedchars=true, texcl=true, frame=single,
        morekeywords={size_t}}



\newenvironment{changemargin}[2]{%
\begin{list}{}{%
\setlength{\topsep}{0pt}%
\setlength{\leftmargin}{#1}%
\setlength{\rightmargin}{#2}%
\setlength{\listparindent}{\parindent}%
\setlength{\itemindent}{\parindent}%
\setlength{\parsep}{\parskip}%
}%
\item[]}{\end{list}}


\usetheme{Copenhagen}
\usepackage{color}

\usecolortheme[RGB={50,100,200}]{structure}

\setbeamertemplate{navigation symbols}{}
\setbeamertemplate{itemize
item}{\scriptsize\raise1.25pt\hbox{\donotcoloroutermaths$\blacktriangleright$}}

\setbeamertemplate{frametitle continuation}[from second]

\title{Воспоминания об stdio.}
\author{Евгений Линский}
\date{}
\usefonttheme[onlymath]{serif}

\setbeamerfont{institute}{size=\normalsize}

\setbeamertemplate{footline} {
  \hbox{%
  \begin{beamercolorbox}[wd=0.08\paperwidth,ht=2.5ex,dp=0ex,left]{title in head/foot}%
  \end{beamercolorbox}%
  \begin{beamercolorbox}[wd=0.46\paperwidth,ht=2.5ex,dp=1ex,center]{title in head/foot}%
    \usebeamerfont{title in head/foot}{C++}%\insertshortauthor %\insertshorttitle
  \end{beamercolorbox}%
  \begin{beamercolorbox}[wd=0.46\paperwidth,ht=2.5ex,dp=1ex,right]{date in head/foot}%
    \usebeamerfont{date in head/foot}
    \insertframenumber{} / \inserttotalframenumber{}\hspace*{2ex}
  \end{beamercolorbox}}%
  \vskip0pt%
}

\newcommand*{\escape}[1]{\texttt{\textbackslash#1}}
\newcommand*{\escapeI}[1]{\texttt{\expandafter\string\csname #1\endcsname}}
\newcommand*{\escapeII}[1]{\texttt{\char`\\#1}}

\sloppy

\begin{document}
\begin{frame}
  \vspace{1cm}
  \large
  \maketitle
  \thispagestyle{empty}
  \vspace{1cm}
  \date{}
\end{frame}

\begin{frame}[fragile]{{\tt stdio}}
\begin{minted}[linenos=true]{cpp}
FILE* f1 = fopen("in.txt",....);// файл на диске
FILE* f2 = stdin; // можно читать с клавиатуры
FILE* f3 = stdout; // можно писать на экран
\end{minted}
FILE --- структура, описывающая абстракцию для ввода-вывода (файл на диске, клавиатура, экран).
Что внутри:
\begin{enumerate}
  \item Дескриптор --- идентификатор (целое число) файла внутри ОС
  \item Промежуточный буфер --- быстрее накопить буфер, а потом за один системный вызов записать его на диск, чем для каждого байта делать отдельный системный вызов
  \item Текущее положение в файле
  \item Индикатор ошибки --- была ли ошибка при последней операции
  \item Индикатор конца файла --- достигнут ли конец файла при последней операции
\end{enumerate}
Напрямую с этими полями не работают, а используют функции stdio.
\end{frame}

\begin{frame}[fragile]{{\tt Текстовые и бинарные файла}}
\begin{enumerate}
  \item На диске всегда байты, меняется только способ их интерпретации
  \item Текстовый формат файла
    \begin{enumerate}
        \item Интерпретируется как последовательность символов. Пример: число 100 записывается не как один байт, а как три символа '1' '0' '0' (3 байта).
        \item Есть спецсимволы: перевод строки, табуляция.
        \item Проблемы: разные кодировки, в том числе для спецсимволов (перевод строки '\escape{n}': Linux --- 10, Windows --- 10 13)
        \item Просто интерпретировать, но большой размер файла.
    \end{enumerate}
  \item Бинарный формат файла
    \begin{enumerate}
        \item Сложные форматы (bmp, wav, elf), для работы нужно описание. Пример: число 100 --- как один байт.
        \item Еще пример. Заголовок: первые 4 байта ширина, вторые четыре байта высота. Данные: три байта RGB с выравниванием.
        \item Сложно интерпретировать, но компактный размер файла.
    \end{enumerate}
\end{enumerate}

\end{frame}

\begin{frame}[fragile]{{\tt stdio}}

FILE --- структура, описывающая абстракцию для ввода-вывода (файл на диске, клавиатура, экран).
\begin{lstlisting}
FILE *f1 = fopen("in.txt",....);// файл на диске
FILE *f2 = stdin; // можно читать с клавиатуры
FILE *f3 = stdout; // можно писать на экран
\end{lstlisting}

stdin, stdout --- глобальные переменные в стандартной библиотеке libc.

\begin{itemize}
  \item getchar -- fgetc(stdin), putchar -- fputc(stdout) -- символы
  \item gets, puts -- строки (fgets, fputs)
  \item printf, scanf -- форматный ввод (fprinf, fscanf)
\end{itemize}

\end{frame}

\begin{frame}[fragile]{{\tt Считать строку}}

\begin{lstlisting}
  char s[100];
  gets(s);
  scanf("%s", s);
\end{lstlisting}
Все ли хорошо?
\end{frame}

\begin{frame}[fragile]{{\tt Считать строку}}

\begin{lstlisting}
  char s[100];

  while (ch != '\n' && i < 99 ) {
    s[i] = getchar();
    i++;
  }
  s[i] = 0;
\end{lstlisting}

\begin{lstlisting}
  gets(s); // Max size!!!
  scanf("%s", s); // Max size!!!
  fgets(s, 99, stdin);
  scanf("%99s", s);
  scanf_s("%99s", name, 100); // Microsoft version
\end{lstlisting}

\end{frame}


\begin{frame}[fragile]{{\tt printf/fprintf/sprintf}}
\begin{lstlisting}
fprintf(stdout, ...); //printf
fscanf(stdin, ....); //scanf
char s1[] = "3  4";
sscanf(s1,"%d %d", &a, &b);
char s2[256];
sprintf(s2, "%d + %d = %d", a, b, c);
\end{lstlisting}
\begin{enumerate}
  \item Все это небыстро, т.к. внутри функции нужно разобрать форматную строку
  \item Технология: функция с переменным числом параметров (см. \emph{va\_arg})
\end{enumerate}
\url{https://en.cppreference.com/w/cpp/io/c/fprintf}
\end{frame}


\begin{frame}[fragile]{{\tt Переменное число аргументов в C (printf)}}
va\_start, va\_arg, va\_end --- макросы.
\begin{minted}[linenos=true]{cpp}
void simple_printf(const char* fmt, ...) {
  va_list args;
  //записать в args адрес следующего за fmt параметра на стеке
  va_start(args, fmt);
  while(*fmt != '\0') {
    if(*fmt=='d') {
      //достать со стека переменную типа int
      int i = va_arg(args, int)
      // здесь должен быть код, который
      // выводит int на экран с помощью putc
    }
    fmt++;
  }
  va_end(args);
}
//Труднообнаруживаемые ошибки
printf("%s", 5);
printf("%d %d", 4); printf("%d", 4, 5);
\end{minted}

\end{frame}



\end{document}




